
% example for dissertation.sty
\documentclass[
  % Replace oneside by twoside if you are printing your thesis on both sides
  % of the paper, leave as is for single sided prints or for viewing on screen.
  oneside,
  %twoside,
  11pt, a4paper,
  footinclude=true,
  cleardoublepage=empty
]{scrbook}

\usepackage{dissertation}
\usepackage[portuguese]{babel}
\usepackage[utf8]{inputenc}
\setcounter{secnumdepth}{3}
\setcounter{tocdepth}{3}

% ACRONYMS -----------------------------------------------------

%import the necessary package with some options
\usepackage[acronym,nonumberlist,nomain]{glossaries}

%enable the following to avoid links from the acronym usage to the list
%\glsdisablehyper

%displays the first use of an acronym in italic
\defglsdisplayfirst[\acronymtype]{\emph{#1#4}}

%the style of the Glossary
\glossarystyle{list}

% set the name for the acronym entries page
\renewcommand{\glossaryname}{Siglas}

%this shall be the last thing in the acronym configuration!!
\makeglossaries

% these could go in an acronyms.tex file, and loaded with:
% \loadglsentries[\acronymtype]{Parts/Definitions/acronyms}
% when using this, you may want to remove 'nomain' from the package options

%% **MORE INFO** %%

%to add the acronyms list add the following where you want to print it:
%\printglossary[type=\acronymtype]
%\clearpage
%\thispagestyle{empty}

%to use an acronym:
%\gls{qps}

% compile the thesis in command line with the following command sequence:
% pdlatex dissertation.tex
% makeglossaries dissertation
% bibtex dissertation
% pdlatex dissertation.tex
% pdlatex dissertation.tex

% ----------------------------------------------------------------

% Title
\titleA{CLAV}
\titleB{Autenticação e integração na plataforma iAP}
%\subtitleA{First Part of Subtitle}
%\subtitleB{Second part of Subtitle} % (if any)

% Author
\author{Octávio Maia}

% Supervisor(s)
\supervisor{Professor José Carlos Ramalho}
%\cosupervisor{The cosupervisor of the thesis}

% University (uncomment if you need to change default values)
% \def\school{Escola de Engenharia}
% \def\department{Departamento de Inform\'{a}tica}
% \def\university{Universidade do Minho}
% \def\masterdegree{Computer Science}

% Date
\date{\myear} % change to text if date is not today

% Keywords
%\keywords{master thesis}

% Glossaries & Acronyms
%\makeglossaries  %  either use this ...
%\makeindex	   % ... or this

% Define Acronyms
% here are the acronym entries
\newacronym{miei}{MIEI}{Mestrado Integrado em Engenharia Informática}
\newacronym{um}{UM}{Universidade do Minho}
\newacronym{dglab}{DGLAB}{Direção-Geral do Livro, dos Arquivos e das Bibliotecas}

%clav
\newacronym{clav}{CLAV}{Projeto M51-CLAV-Arquivo digital: Plataforma modular de classificação e avaliação da informação pública}
\newacronym{iap}{iAP}{Interoperabilidade na Administração Pública}
\newacronym{ap}{AP}{Administração Pública}
\newacronym{lc}{LC}{Lista Consolidada}
\newacronym{cc}{CC}{Cartão de Cidadão}
\newacronym{bi}{BI}{Bilhete de Identidade}
\newacronym{cmd}{CMD}{Chave Móvel Digital}
\newacronym{ama}{AMA}{Agência para a Modernização Administrativa}
\newacronym{pki}{PKI}{Public key infrastucture}
\newacronym{saml}{SAML}{Security Assertion Markup Language}
\newacronym{stork}{STORK}{Secure identity across borders linked}
\newacronym{fa}{FA}{Fornecedor de Autenticação}
\newacronym{nic}{NIC}{Número Identificação Civil}

%%baseado com jwt
\newacronym{jwt}{JWT}{JSON Web Token}
\newacronym{hmac}{HMAC}{Hash-based Message Authentication Code}
\newacronym{rsa}{RSA}{Rivest-Shamir-Adleman}
\newacronym{sha}{SHA}{Secure Hash Algorithm}

%%bcrypt
\newacronym{owasp}{OWASP}{Open Web Application Security Project}

%%random
\newacronym{http}{HTTP}{Hyper Text Transfer Protocol}
\newacronym{https}{HTTPS}{Hyper Text Transfer Protocol Secure}
\newacronym{ssl}{SSL}{Secure Sockets Layer}
\newacronym{tls}{TLS}{Transport Layer Security}
\newacronym{md5}{md5}{Message-Digest algorithm 5}
\newacronym{api}{API}{Application programming interface}
\newacronym{cpu}{CPU}{Central Processing Unit}
\newacronym{gpu}{GPU}{Graphics Processing Unit}
\newacronym{fpga}{FPGA}{Field-programmable gate array}
\newacronym{asic}{ASIC}{Application-specific integrated circuit}
\newacronym{usenix}{USENIX}{The Advanced Computing Systems Association}
\newacronym{posix}{POSIX}{Portable Operating System Interface}
\newacronym{ocsp}{OCSP}{Online Certificate Status Protocol}
\newacronym{xml}{XML}{Extensible Markup Language}
\newacronym{oasis}{OASIS}{Organization for the Advancement of Structured Information Standards}
\newacronym{sso}{SSO}{Single-Sign On}
\newacronym{itu}{ITU}{International Telecommunication Union}
\newacronym{eeprom}{EEPROM}{Electrically-Erasable Programmable Read-Only Memory}
\newacronym{jvm}{JVM}{Java Virtual Machine}
\newacronym{puk}{PUK}{Pin Unlock Key}
\newacronym{des}{DES}{Data Encryption Standard}
\newacronym{tdes}{TDES}{Triple Data Encryption Standard}
\newacronym{mac}{MAC}{Message Authentication Code}
\newacronym{pkcs}{PKCS}{Public-Key Cryptography Standards}
\newacronym{cap}{CAP}{Chip Authentication Program}
\newacronym{cors}{CORS}{Cross-Origin Resource Sharing}
\glsaddall[types={\acronymtype}]

\ummetadata % add metadata to the document (author, publisher, ...)

\begin{document}
	% Cover page ---------------------------------------
	\umfrontcover	
	\umtitlepage
	
	% Add acknowledgements ----------------------------
    %\input{0_acknowledgements.tex}

	% Resumo ------------------------
	\cleardoublepage
	\chapter*{Resumo}

Existe uma preocupação cada vez maior em relação à quantidade de informação gerada e recebida por diversas instituições. Não só no que toca ao consumo excessivo de papel, como também a gestão de grandes quantidades de informação.

Com o intuito de simplificar a gestão documental, o governo tem desenvolvido diversas estratégias. Nomeadamente, na \gls{ap}, com base em normas e orientações provenientes da Comissão Europeia.

O \gls{clav}, surge como uma dessas estratégias. Este visa a classificação e a avaliação de toda a documentação que circula na \gls{ap} portuguesa, utilizando um referencial comum que permite o desenvolvimento de instrumentos de natureza transversal a aplicar em contexto organizacional.

Esta dissertação tem como objetivo primário, a integração dos serviços da plataforma \gls{auth.gov} no M51-CLAV, bem como a criação de estratégias apropriadas de registo e login que estejam de acordo com as especificações fornecidas pelo Autenticação.Gov.

Após realizada a integração com a plataforma Autenticação.Gov, serão desenvolvidos mecanismos capazes de gerir os diversos utilizadores do M51-CLAV.

	% Add abstracts (en,pt) ---------------------------
	\chapter*{Abstract}

There's a ever growing worry about the quantity of information that's generated and received by multiple institutions. Not only related to the excessive ammount of paper consumed, but as well as the management of such big quantities of information.

With the purpose of simplifying the process of managing such documents, the portuguese government has been developing various strategies. Namely, in the Public Administration, with directives provided by the European Comission.

The \gls{clav} project comes up as one of those strategies. This allows to classify and evaluate all the documentation that circulates in the Portuguese Public Administration, using a common referential, that allows for the development of instruments of transversal nature that are applied in an organizational context.

This dissertation has as its primary objective, the integration of the services provided by the \emph{Autenticação.Gov} platform in the \gls{clav} project, as well as the creation of appropriate strategies of user management, authentication of requests refering to the public API, as well as  backend authentication and overall security.

\vspace{5cm}
\keywords{CLAV, Autenticação.Gov, Authentication, Security}
	
	% Summary Lists ------------------------------------
	\tableofcontents
	\listoffigures
	\listoftables
	%\lstlistoflistings
	\listofabbreviations
	%\printglossary[type=\acronymtype]
	\clearpage
	\thispagestyle{empty}
	\pagenumbering{arabic}
	
	% CHAPTER - Introduction -------------------------
    \chapter{Introdução}

\section{Enquadramento}

Os processos burocráticos, administrativos e logísticos a que hoje as instituições estão sujeitas levam à produção de grandes quantidades de informação que, na sua maioria devem estar disponíveis para consulta a longo-termo, não só por motivos legais, mas também por questões associadas à preservação de dados arquivísticos. 

A grande quantidade de informação produzida desencadeia longos e complexos processos que arrecadam para as instituições despesas elevadas e impacto ambiental significativo. A incomportabilidade deste contexto, torna a desmaterialização dos processos fundamental, que se identifica como uma medida enquadrada no eixo estratégico do desenvolvimento sustentável.[1]

O Governo tem neste sentido, desenvolvido estratégias para a transformação digital, nomeadamente, na Administração Pública (AP), tendo por base normas e orientações provenientes da Comissão Europeia. Entre outros, pretende-se a redução do consumo de papel na mesma [2], através da desmaterialização de processos, da promoção da adoção de sistemas de gestão documental eletrónica ou outros, bem como a digitalização de documentos destinados a ser arquivados. 

Uma das medidas previstas foi a adoção de processos de «classificação, avaliação e seleção de informação, tendo em consideração, sempre que possível, os princípios de uma Macroestrutura Funcional (MEF) e a Avaliação Supra-Institucional da Informação Arquivística (ASIA)».[3][4]

Assim sendo, surgiu o “Projeto M51-CLAV-Arquivo digital: Plataforma modular de classificação e avaliação da informação pública”, projeto nacional financiado pela Direção-Geral do Livro, dos Arquivos e das Bibliotecas (DGLAB) em colaboração com a Universidade do Minho, no âmbito do Aviso N.º 02/SAMA2020/2016 (fazer referencia), para dar cumprimento à Medida 51 do Simplex + “Arquivo digital”. 

O Simplex é um programa do governo que visa a simplificação legislativa e administrativa, e a modernização dos serviços públicos. Em particular, com o objetivo de tornar a Administração Pública mais eficiente surgiu o “Arquivo Digital” do Ministério da Cultura. Este pretende utilizar instrumentos transversais de gestão da informação, com o fim de classificar e avaliar todos os documentos produzidos e recebidos nos organismos públicos. 

Estes instrumentos serão disponibilizados por uma plataforma modular de serviços partilhados, com a possibilidade de integração com os sistemas de informação existentes em qualquer organismo, sendo que esta plataforma também permite a desmaterialização dos procedimentos, atualmente obrigatórios, para se poder eliminar documentação em papel no Estado.

Através de um projeto colaborativo que envolveu grande parte dos organismos da AP, Central e Local, foi originada a Lista Consolidada (LC) para a classificação e avaliação da informação pública. Esta LC foi desenvolvida usando uma abordagem supra-institucional e funcional, cujo valor maior é a interoperabilidade semântica, viabilizada pela criação de uma linguagem comum e transversal à AP, que resultou, entre outros aspetos, na criação de códigos de classificação comuns.

Este trabalho enquadra-se no Projeto CLAV, nas suas terceiras e quarta fase de desenvolvimento, na implementação de uma aplicação Web para a gestão e manutenção da plataforma CLAV. 

%\cleardoublepage
%\section{Motivação}
%escrever aqui cenas da autenticacao e seguranca

\cleardoublepage
\section{Objetivos}
%ver o que ja foi enviado

A realização desta dissertação tem como objetivo os seguintes pontos:

\begin{itemize}
    \item Estudo e interpretação dos serviços da plataforma Autenticação.Gov e respetivos requisitos.
    \item Integração e desenvolvimento de métodos apropriados de registo e login de utilizadores, bem como a gestão dos mesmos, utilizando a ferramenta Autenticação.Gov.
    \item Avaliação dos métodos gerados, garantindo assim uma correta implementação dos mesmos na plataforma M51-CLAV.
\end{itemize}

\cleardoublepage
\section{Metodologia}
%ver o que ja foi enviado

A metodologia para a realização desta dissertação segue os seguintes passos:

\begin{itemize}
    \item Leitura e análise inicial do projeto.
    \item Estudo da ferramenta Autenticação.Gov e requisitos por ela
impostos.
    \item Desenvolvimento de um protótipo com funcionalidades
semelhantes ou idênticas à esperada.
    \item Implementação na plataforma M51-CLAV.
\end{itemize}

	% CHAPTER - State of the Art ---------------------
    \chapter{Estado da Arte}

Neste capítulo é apresentado o estado da arte relativo ao contexto desta dissertação.

	% CHAPTER - Problem and Challenges ---------------
    %\input{sections/3_desafios.tex}
	
	% CHAPTER - Contribution -------------------------
    %\input{sections/4_contribuicao.tex}

	% CHAPTER - Application -------------------------
    %\input{sections/5_aplicacao.tex}

	% CHAPTER - Conclusion/Future Work --------------
	\chapter{Conclusões e Trabalho futuro}

\section{Conclusões}
\section{Trabalho futuro}
			
	\bookmarksetup{startatroot} % Ends last part.
	\addtocontents{toc}{\bigskip} % Making the table of contents look good.
	%\cleardoublepage

	%- Bibliography (needs bibtex) -%
	\bibliography{dissertation}

	% Index of terms (needs  makeindex) -------------
	%\printindex
	
	% APPENDIX --------------------------------------
	\umappendix{Appendix}
	
	% Add appendix chapters
    \input{sections/apendice.tex}

	% Back Cover -------------------------------------------
	%\umbackcover{
	%NB: place here information about funding, FCT project, etc in which %the work is framed. Leave empty otherwise.
	%}


\end{document}
