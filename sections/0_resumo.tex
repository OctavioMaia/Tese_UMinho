\chapter*{Resumo}

Existe uma preocupação cada vez maior em relação à quantidade de informação gerada e recebida por diversas instituições. Não só no que toca ao consumo excessivo de papel, como também a gestão de grandes quantidades de informação.

Com o intuito de simplificar a gestão documental, o governo tem desenvolvido diversas estratégias. Nomeadamente, na \gls{ap}, com base em normas e orientações provenientes da Comissão Europeia.

O \gls{clav}, surge como uma dessas estratégias. Este visa a classificação e a avaliação de toda a documentação que circula na Administração Pública portuguesa, utilizando um referencial comum que permite o desenvolvimento de instrumentos de natureza transversal a aplicar em contexto organizacional.

Esta dissertação tem como objectivo primário, a integração dos serviços da plataforma \emph{Autenticação.Gov} no \gls{clav}, bem como a criação de estratégias apropriadas de gestão de utilizadores, autenticação de pedidos referentes à \emph{API} pública disponibilizada, autenticação no backend e segurança da aplicação.

\vspace{5cm}
\keywords{CLAV, Autenticação.Gov, Autenticação, Segurança}