\chapter{Introdução}

\section{Enquadramento}

Os processos burocráticos, administrativos e logísticos a que hoje as instituições estão sujeitas levam à produção de grandes quantidades de informação que, na sua maioria devem estar disponíveis para consulta a longo-termo, não só por motivos legais, mas também por questões associadas à preservação de dados arquivísticos. 

A grande quantidade de informação produzida desencadeia longos e complexos processos que arrecadam para as instituições despesas elevadas e impacto ambiental significativo. A incomportabilidade deste contexto, torna a desmaterialização dos processos fundamental, que se identifica como uma medida enquadrada no eixo estratégico do desenvolvimento sustentável.\cite{artigoOECD}

O Governo tem neste sentido, desenvolvido estratégias para a transformação digital, nomeadamente, na Administração Pública, tendo por base normas e orientações provenientes da Comissão Europeia. Entre outros, pretende-se a redução do consumo de papel na mesma \cite{papZero}, através da desmaterialização de processos, da promoção da adoção de sistemas de gestão documental eletrónica ou outros, bem como a digitalização de documentos destinados a ser arquivados. 

Uma das medidas previstas foi a adoção de processos de «classificação, avaliação e seleção de informação, tendo em consideração, sempre que possível, os princípios de uma Macroestrutura Funcional (MEF) e a Avaliação Supra-Institucional da Informação Arquivística (ASIA)».\cite{congressoBAD}\cite{confGIA}

Assim sendo, surgiu o “Projeto M51-CLAV-Arquivo digital: Plataforma modular de classificação e avaliação da informação pública”, projeto nacional financiado pela \gls{dglab} em colaboração com a \gls{um}, no âmbito do Aviso N.º 02/SAMA2020/2016\cite{compete2020}, para dar cumprimento à Medida 51 do Simplex + “Arquivo digital”. 

O Simplex é um programa do governo que visa a simplificação legislativa e administrativa, e a modernização dos serviços públicos. Em particular, com o objetivo de tornar a Administração Pública mais eficiente surgiu o “Arquivo Digital” do Ministério da Cultura. Este pretende utilizar instrumentos transversais de gestão da informação, com o fim de classificar e avaliar todos os documentos produzidos e recebidos nos organismos públicos. 

Estes instrumentos serão disponibilizados por uma plataforma modular de serviços partilhados, com a possibilidade de integração com os sistemas de informação existentes em qualquer organismo, sendo que esta plataforma também permite a desmaterialização dos procedimentos, atualmente obrigatórios, para se poder eliminar documentação em papel no Estado.

Através de um projeto colaborativo que envolveu grande parte dos organismos da Admi-nistração Pública, Central e Local, foi originada a \gls{lc} para a classificação e avaliação da informação pública. Esta \gls{lc} foi desenvolvida usando uma abordagem supra-institucional e funcional, cujo valor maior é a interoperabilidade semântica, viabilizada pela criação de uma linguagem comum e transversal à Administração Pública, que resultou, entre outros aspetos, na criação de códigos de classificação comuns.

Este trabalho enquadra-se no Projeto \gls{clav}, nas suas terceiras e quarta fase de desenvolvimento, na implementação de uma aplicação Web para a gestão e manutenção da plataforma \gls{clav}. 

%\cleardoublepage
%\section{Motivação}
%escrever aqui cenas da autenticacao e seguranca

%\cleardoublepage
\section{Objetivos}

A realização desta dissertação tem como objetivo os seguintes pontos:

\begin{itemize}
    \item Estudo e interpretação dos serviços da plataforma Autenticação.Gov e respetivos requisitos.
    \item Integração e desenvolvimento de métodos apropriados de registo e login de utilizadores, bem como a gestão dos mesmos, utilizando a ferramenta Autenticação.Gov.
    \item Criação de métodos apropriados para autenticação de pedidos via a API pública disponibilizada.
    \item Avaliação dos métodos gerados, garantindo assim uma correta implementação dos mesmos na plataforma \gls{clav}.
\end{itemize}

\cleardoublepage
\section{Metodologia}

A metodologia para a realização desta dissertação segue os seguintes passos:

\begin{itemize}
    \item Leitura e análise inicial do projeto.
    \item Estudo das ferramentas já implementadas.
    \item Estudo da ferramenta Autenticação.Gov e requisitos por ela impostos.
    \item Desenvolvimento de um protótipo com funcionalidades semelhantes ou idênticas à esperada.
    \item Implementação na plataforma \gls{clav}.
\end{itemize}

\section{Síntese}

Neste capítulo foi realizada uma breve introdução e enquadramento desta dissertação no projeto \gls{clav}, listando os objetivos propostos durante o desenvolvimento da mesma, bem como a metodologia adoptada.

No capítulo seguinte irá ser documentado o estado da arte atual, sendo focado em especial atenção o Autenticação.Gov, sendo descrito o seu workflow, funcionamento e especificação técnica. Além deste, será também mencionada a especificação do Cartão de Cidadão e da proteção da API de dados disponibilizada ao público.